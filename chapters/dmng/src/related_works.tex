\section{Related work}

Several research works have highlighted the importance of sandboxing
and isolation techniques in modern software~\cite{kim2013practical,
  lsm_fra, berman1995tron, RLBox, seapp, ALBANESE2020934}. Indeed,
sandboxing plays a key role in many plaftorms (e.g., Linux, Windows,
iOS, Android), and is integrated in widely used software such as
browsers (e.g., Chrome~\cite{chromium-sandbox},
Firefox~\cite{firefox-sandbox}), service managers (e.g.,
Systemd~\cite{systemd-sandbox}) and document viewers (e.g.,
Acrobat~\cite{acrobat-sandbox}).

With specific reference to the cloud scenario, many recent proposals
have investigated the use of sandboxing to mitigate
vulnerabilities~\cite{natisand, cage4deno, enhance-wasm-sandbox,
  sanddriller-staicu, staicu2021bilingual, binwrap, zimmermann-risks,
  npm-malicious-update}. In {\em NatiSand}~\cite{natisand} and {\em
  Cage4Deno}~\cite{cage4deno} the authors modify the Deno runtime to
control the permissions available to applications running native
code. {\em BinWrap}~\cite{binwrap} proposes similar measures to
restrict the permissions available to Node.js native add-ons.  {\em
  SandDriller}~\cite{sanddriller-staicu} describes an approach based
on dynamic analysis for detecting sandbox escape vulnerabilities for
Node.js applications. Zimmermann et al.~\cite{zimmermann-risks} and
Ferreira et al.~\cite{npm-malicious-update} study the risks associated
with vulnerable or malicious third-party dependencies and propose
possible install (and update) time countermeausures. In general, all
the previous proposals address the issues associated with a specific
runtime ecosystem. Conversely, we aim to secure applications
independently of their build toolchain or runtime.

Virtual machines and containers are two fundamental technologies in
modern cloud architectures. Both permit to virtualize resources and
execute applications in an isolated environment. Virtual machines
ensure stronger security guarantees at the cost of higher resource
utilization with respect to containers. The main reason is that
applications executed in separate virtual machines have a distinct set
of resources and do not share the same
kernel~\cite{casola-optimization, CASOLA2018235}.  With specific
reference to our scenario, both these technologies are associated with
coarse granularity. Indeed, when working with them developers grant
the application access to volumes rather than single resources. So, we
provide a complementary approach to enable the introduction of
fine-grained, per-resource access rules.

Modern industrial platforms like Cilium~\cite{cilium} and
Falco~\cite{falco} rely on eBPF as the primary means to enforce
security policies in cloud applications. Cilium provides networking,
observability, and security functions for container workloads, while
Falco implements a threat detection engine for clusters. Both
solutions are enterprise-oriented, hence the developer may find
difficult to set up fine-grained policies leveraging them. Moreover,
as already mentioned in the chapter, the performance of eBPF-based
solutions is associated with large variability when fine-grained rules
are used~\cite{falco-overhead}. Therefore, we propose to complement
these solutions by assisting the developer in the generation of least
privilege security policies and using recent sandboxing technologies
like Landlock, to reduce the overhead and strengthen the security
boundary of the application.

%%% Local Variables:
%%% mode: latex
%%% TeX-master: "../main"
%%% End:

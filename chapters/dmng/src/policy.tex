\section{Policy}\label{sect:policy}

In this section we present the structure of the policy, explaining how
it can be customized. We also discuss aspects such as coverage and
effectiveness.

\paragraph*{Policy structure} The policy obtained from \dmng is a JSON
file structured as a list of objects as shown in
Listing~\ref{lst:policy_file}. For each of them, a field {\em
  policy\textunderscore name} identifies the application component the
policy applies to, while the sections {\em read}, {\em write} and {\em
  exec} are used to configure the related permissions. The structure
of the policy is flexible, and for each object only the field {\em
  policy\textunderscore name} is required. Since the policy implements
a \mbox{{\em default-deny}} model, an object that does not list any
section in its body has no runtime permission, hence the corresponding
component cannot access any file system
resource. Listing~\ref{lst:policy_file} shows an example in which a
component called {\em filter} is granted execution access to the Awk
program (together with its shared libraries) to process the
\mbox{read-only} {\em users.csv} dataset.

\paragraph*{Policy customization} \dmng provides a CLI interface that
work simultaneously with multiple data sources to produce the list of
requirements. By default, it implements the logic to automatically
merge the requirements collected with ptrace and the eBPF
programs. Moreover, it permits to customize the policy interactively,
adding, changing and removing requirements. For instance, it allows to
delete requirements based on the permission mask (e.g., {\tt
  r\textunderscore x}, {\tt r--}), or change the permission associated
with all the requirements that match a given path regex (e.g., {\tt
  /usr/bin/libnet.*}).  \newline A useful feature implemented by \dmng
is {\em permission pruning}. This function takes advantage of the
structure of the Directory Tree~\cite{hier} to help the developer
lower the number of requirements in the policy by reducing the
granularity of permissions. The reduction in granularity is based on a
{\em pruning goal} set by the developer that represents the desirable
maximum number of policy rules associated with an application
component. To implement this feature, \dmng first uses the policy
template to build a trie (or prefix tree), then starts pruning its
branches iteratively following a best effort approach, until the
pruning goal is achieved.  The rationale is that there are areas of
the file system in which fine granularity brings strong security
guarantees (e.g., {\tt /lib}), but there are also many other areas
where fewer rules make the policy more concise without affecting
security (e.g., {\tt /share}). So, it is important to consider
contextual information about the current prefix path to guide the
pruning process.  Moreover, we want to comply with the {\em write xor
  execute} memory protection policy whereby every file may be either
writable or executable, but not both. Thus, limiting the propagation
of potentially insecure configurations (e.g., no dynamic library
stored in {\tt /lib} must be writable and executable by the
application).  After the pruing process terminates, the changes to the
template are audited by the developer, and can be committed or
discarded.


\begin{lstlisting}[
  caption=Example of JSON file with single policy,
  float=tp,
  floatplacement=tbp,
  language=json,
  label=lst:policy_file,
  frame=none,
  aboveskip=-0.3em,
]
{
  "policies": [{
      "policy_name": "filter",
      "read": [
        "/lib/x86_64-linux-gnu/libsigsegv.so.2",
        "/lib/x86_64-linux-gnu/libreadline.so.8",
        "/lib/x86_64-linux-gnu/libmpfr.so.6",
        "/lib/x86_64-linux-gnu/libgmp.so.10",
        "/lib/x86_64-linux-gnu/libm.so.6",
        "/lib/x86_64-linux-gnu/libc.so.6",
        "/lib/x86_64-linux-gnu/libtinfo.so.6"',
        "/lib64/ld-linux-x86-64.so.2",
        "/usr/bin/awk",
        "users.csv"
      ],
      "exec": [
        "/lib/x86_64-linux-gnu/libsigsegv.so.2",
        "/lib/x86_64-linux-gnu/libreadline.so.8",
        "/lib/x86_64-linux-gnu/libmpfr.so.6",
        "/lib/x86_64-linux-gnu/libgmp.so.10",
        "/lib/x86_64-linux-gnu/libm.so.6",
        "/lib/x86_64-linux-gnu/libc.so.6",
        "/lib/x86_64-linux-gnu/libtinfo.so.6"',
        "/lib64/ld-linux-x86-64.so.2",
        "/usr/bin/awk"
      ]
  }]
}
\end{lstlisting}

\paragraph*{Coverage and effectiveness}

To generate the policy templates, \dmng registers the activity
performed by the application while a set of test cases is
executed. This approach is similar to the one proposed by the Slim
toolkit~\cite{slimtoolkit} to identify the dependencies of a container
and minify its image, and to the one followed by the Google Sandbox2
utility~\cite{sandbox2} to retrieve the requirements of programs
distributed as ELF files. However, test-based policy generation can be
subject to coverage issues if the set of test cases is not
exhaustive. Another aspect worth mentioning is that applications
poorly structured may benefit less from the isolation properties
provided by sandboxing. Indeed, on a traditional Unix operating
system, components executed within the same thread will inevitably
share the same policy. Since \dmng supports the sandboxing of
components with a per-thread policy, we recommend to leverage this
function and execute potentially vulnerable components in dedicated
compartments.

%%% Local Variables:
%%% mode: latex
%%% TeX-master: "../main"
%%% End:

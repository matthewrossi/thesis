\section{Introduction}

Cloud applications are often built of many components, each
implementing a specific set of business requirements. Components
naturally evolve, their requirements may change, and as the overall scale
of the application grows it can be challenging to ensure a high
security standard. Many factors contribute to making this objective
hard to achieve; the most common ones are: the presence of buggy
components, the reliance on potentially vulnerable native libraries
written with memory unsafe languages, and broad access to system
resources.

Several research works have investigated the scenario (e.g.,
\cite{staicu2021bilingual, binwrap,natisand, zimmermann-risks}). For
instance, Staicu et al.~\cite{staicu2021bilingual} explain the
security problems that arise when web applications interact with
native extensions. BinWrap~\cite{binwrap} and NatiSand~\cite{natisand}
analyze recent vulnerabilities and integrate new security
measures in the Node.js and Deno runtimes to improve isolation and
limit access to confidential resources. Lastly, 
Zimmermann et al.~\cite{zimmermann-risks} present an extensive study of third-party
dependencies, finding that large parts of the entire web ecosystem can
be impacted by security issues even when individual packages are
vulnerable or they include malicious code on purpose.

Recently, industry standards have also emerged to encourage
organizations to proactively include new security measures. A notable
example is the NIST SP 800-190~\cite{nist-sp800-190}, which focuses on
environments that adopt microservices, containers and Kubernetes. The
production environment is given particular attention, with the goal of
finding and stopping malicious threats in real time.  The directive
clearly indicates that there is a need for policies to defend against
vulnerabilities that could lead to disruptions, such as modification
of important files. The same regulation also provides instructions to
prevent the tampering of the file system, prompting that applications
and containers must be run with a set of permissions as minimal as
possible, namely following the {\em least privilege} principle.

Powered by the recent eBPF kernel technology, frameworks like
Tetragon~\cite{tetragon} and Falco~\cite{falco} have been proposed to
monitor cloud applications, identify unexpected security-relevant
events, and act on them (e.g., denying access). While effective, they
tend to introduce non-negligible overhead when fine-grained policy
rules are used~\cite{falco-overhead}.  We argue that the combined use
of classical operating system access control and sandboxing mechanisms
may lead to a more resource efficient solution to isolate application
components.  For instance, the recent Landlock LSM~\cite{landlock}
permits a process to restrict itself ensuring strong security
guarantees with minimum performance footprint.  Furthermore, the
integration of Landlock, or an equivalent security mechanism, in cloud
applications significantly mitigates the risk associated with the
exploitation of a vulnerability, as the amount of resources available
to a potential attacker is greatly reduced.

Unfortunately, to benefit from this protection developers must obtain
a policy that clearly states the resources an application component
must be granted access to, and the related permissions. This is far
from trivial in the case of complex applications. Indeed, the list of
resources can vary based on the production environment, or be subject
to changes when different inputs are provided. All these reasons hold
back the potential of sandboxing, and cloud applications may solely
rely on the coarse isolation provided by the virtual machine or the
container in which the application is executed.

\paragraph*{Our contribution}
In this chapter we propose a new approach to systematically integrate
fine-grained sandboxing in cloud applications that is aligned with the
current regulations and best practices. In detail, we provide an
intuitive, open-source solution to retrieve all the file system
resources required by an application component, to build and customize
least privilege policies. We then showcase how policies are used to
sandbox programs with Landlock, mitigating the impact of severe
CVEs. The approach we propose is flexible and does not depend on the
toolchain leveraged to build each application component. The
experiments showcase the minimal performance footprint at runtime, and
highlight the ability to monitor the application without affecting its
execution state.

%%% Local Variables:
%%% mode: latex
%%% TeX-master: "../main"
%%% End:

\section{Motivation}\label{sect:motivation}

This Section explains the threat model and the motivation.

\subsection{Threat model}\label{subsect:threat-model}

We assume that the code part of the cloud application (including
native code executed by it) is trusted and not malicious, but
potentially affected by vulnerabilities due to bugs.  In this
scenario, an attacker may leverage web interfaces or programmatic APIs
to send the application malicious payloads with the goal of exploiting
such vulnerabilities. This attack vector may leave the application
exposed to, e.g., arbitrary file read and write, file system
compromise, and execution of arbitrary programs; all of which can lead
to an inconsistent state of the application and its volumes. We aim at
the creation of fine-grained, component-specific policies that are
used by developers to gradually introduce sandboxing, mitigating the
impact of vulnerabilities. This approach is complementary to the use
of containers, as a compromised process running in a container can
still damage all the resources available to it.

\subsection{Dependency identification}

An important use case for cloud applications is represented by
services that handle media resources such as videos, photos, and
audio. These applications typically rely on an extensive set of
editing libraries and codecs to perform a number of operations like
crop, scale, introduction of effects, format conversion, and
compression. This software is usually available as dynamic libraries
that are loaded and executed depending on the type of operation to be
performed, on the input source, and on the hardware support
available. A solution able to automatically collect all the resources
used by a component for a set of test cases can significantly help the
developer detecting and isolating the dependencies of the application.

\subsection{Mitigation of bugs}

Open source libraries and programs are used extensively while
developing cloud applications. Examples include database drivers,
media processing libraries, and encoding utilities.  This software is
often trusted, but it may be subject to vulnerabilities as explained
in Section~\ref{subsect:threat-model}.  When vulnerable third-party
code is executed by the application, it can be targeted and
compromised by an attacker who sends malicious payloads, as described
in~\cite{sanddriller-staicu, cage4deno, npm-malicious-update}.
Popular cases are the Server-Side Request Forgery and Arbitrary File
Read vulnerabilities found in FFmpeg and exploited against
TikTok~\cite{ffmpeg-exploit-example}, and the Remote Command Execution
vulnerability found in ExifTool and exploited against
GitLab~\cite{exiftool-exploit-example}. Other examples include:
1)~CVE-2020-24020, CVE-2022-2566, CVE-2020-2499 associated with video
processing software, 2)~CVE-2022-1292, CVE-2022-2068, CVE-2022-2274
related to cryptographic software, and 3)~CVE-2021-4118,
CVE-2021-37678, CVE-2022-0845 targeting machine learning software.

With our approach we aim to support the progressive introduction of
fine-grained policies that are used to sandbox the application or any
of its components, making harder for an attacker to tamper the file
system, corrupt data, or exfiltrate sensitive information such as
private keys or database entries.

\subsection{Performance and usability}

Modern cloud applications are often deployed as Kubernetes Pod
instances and executed in one or more containers. Kubernetes provides
several tools to improve the security and isolation of Pods. Relevant
to this scenario are the support for RBAC policies and the
availability of restricted Seccomp profiles. RBAC policies by default
permit to define access of human users, however, they can also be used
to govern the behavior of software resources through {\em service
  accounts}. Unfortunately, these policies can only restrict access to
Kubernetes APIs, therefore they are not suitable for fine-grained
restriction of permissions at an application component level. The same
limitation is shared with Seccomp profiles, which can limit the kernel
interface available to the cloud application, but are only applied at
container level~\cite{k8s-seccomp} and cannot operate depending on the
specific requested resource~\cite{jia2023programmable}.

Recent solutions based on eBPF, like Tetragon and Falco, enable the
introduction of fine-grained policy rules to overcome the previous
limitations. To this end, they load into the kernel dedicated filters
which are run system-wide every time a security event like the opening of a file
or the execution of a program occurs. Based on the content of the
policy, and therefore the filters, these frameworks can grant or deny
a particular action, effectively restricting the privileges available
to a cloud application. However, as mentioned in Falco's
documentation~\cite{falco-overhead}, the main problem associated with
this approach is that performance overhead can have a large
variability. This is mainly due to the fact that filters are evaluated
every time a certain hook point (e.g., a syscall) is triggered, and
that many hooks may need to be controlled to enforce a given security
policy. Furthermore, these frameworks do not assist the developer in
the generation of application-specific policies.

With our proposal we aim at giving the developer a complementary
approach to secure applications and limit the file system resources
accessible to them.  Our idea is that the developer can benefit from
the advantages brought by technologies like Landlock (i.e., strong
security guarantees, low overhead), while at the same time rely on
eBPF-based technologies, but only to monitor a restricted subset of
important security events.

%%% Local Variables:
%%% mode: latex
%%% TeX-master: "../main"
%%% End:

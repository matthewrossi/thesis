\section{Document structure}

The thesis is organized in six chapters.

\paragraph*{Chapter~\ref{chap:intro}} illustrates the document
structure and the publications that set the basis for this thesis.

\paragraph*{Chapter~\ref{chap:seapp}} describes \seapp~\cite{seapp}, a
proposal that provides developers with a mechanism to isolate the
internal components of Android apps and regulate their permissions on
a per-component basis. This is achieved by first executing components
in dedicated processes, and then restricting access to the app and
system resources with ad hoc \sel policies. Specifically, it is
possible to declare rules to regulate access to both the internal app
storage and system services, which are otherwise shared between all
the application components, including third-party libraries the app
depends upon. This is a crucial step to provide strong user privacy
guarantees despite relying on third-party code to achieve monetization
and fast-paced development of Android applications. The prototype
implements a patch to the Android Open Source Platform (AOSP) to
demonstrate the feasibility, effectiveness and efficiency of the novel
approach.

\smallskip
\noindent The chapter is organized as follows.

\begin{compactitem}
    \item Section~\ref{sect:seapp_introduction} provides an overview
     of how the security architecture of mobile systems evolved with
     the maturity of the ecosystem.
    \item Section~\ref{sect:seapp_andro_sec} introduces the techniques
     currently enforcing access control in Android.
    \item Section~\ref{sect:seapp_motiv} presents the motivation for
     the introduction of intra-app isolation in Android. Specifically,
     a set of use cases is used to showcase the security measures
     introduced by \seapp.
    \item Section~\ref{sect:seapp_lang} details the \seapp policy
     module syntax, and its constraints to ensure proper integration
     of app and system policies.
    \item Section~\ref{sect:seapp_config} illustrates the policy 
     configuration files of \sea and how to use them in \seapp.
    \item Section~\ref{sect:seapp_implementation} discusses the
     changes the \seapp implementation introduced in Android platform.
    \item Section~\ref{sect:seapp_performance} presents the
     experimental evaluation, in which we measure both the
     installation time and runtime overhead introduced by \seapp.
    \item Section~\ref{sect:seapp_relwork} discusses the major
     differences between \seapp and other literature proposals.
    \item Section~\ref{sect:seapp_conclusions} concludes the chapter.
\end{compactitem}
\medskip

% TODO: add summary of dmng
\smallskip
\noindent The chapter is organized as follows.
\begin{compactitem}
    \item Section~\ref{dmng:sect:introduction} presents the challeges
     of securing cloud applications and briefly discusses the state of
     the art to identify opportunities for improvement.
    \item Section~\ref{sect:motivation} discusses the objectives of
     the solution together with its requirements and trust
     assumptions.
    \item Section~\ref{sect:overview} illustrates an overview of the
     approach by highlighting its integration with both staging and
     production environments.
    \item Section~\ref{sect:cloud-instrum} presents two different
     techniques to instrument cloud applications and trace their
     activity.
    \item Section~\ref{dmng:sect:policy} discusses the generation of
     access control policies from activity traces.
    \item Section~\ref{sect:sandbox} details how we implement the
     lightweight enforcement of the policies.
    \item Section~\ref{dmng:sect:exp} presents an empirical
     evaluation of the mitigation capabilities of our approach and
     its performance.
    \item Section~\ref{dmng:sect:related-work} discuss proposals from
     the literature and how they compare to ours.
    \item Section~\ref{dmng:sect:conclusions} concludes the chapter.
\end{compactitem}
\medskip

% TODO: add summary of wasm
\smallskip
\noindent The chapter is organized as follows.
\begin{compactitem}
    \item Section~\ref{intro} introduces WebAssembly and the
     challenges WebAssembly runtimes are facing with restricting
     access to system resources.
    \item Section~\ref{sect:wasm:threat-model} models the threat
     an attacker can pose to the implementations of WASI in
     WebAssembly runtimes.
    \item Section~\ref{design} details the design of current
     WASI implementations and proposes an alternative approach.
    \item Section~\ref{sec:exp} evaluates the overhead introduced by
     our novel approach in two different WebAssembly runtimes.
    \item Section~\ref{wasm:rel_works} discusses the related work.
    \item Section~\ref{sect:wasm:conclusions} concludes the chapter.
\end{compactitem}
\medskip

% TODO: add summary of natisand
\smallskip
\noindent The chapter is organized as follows.
\begin{compactitem}
    \item Section~\ref{sect:introduction} presents the development of
     web applications with the use of JavaScript runtimes and their
     limitations with regard to controlling native code access to
     system resources.
    \item Section~\ref{sectbackground} overviews the structure of
     moderm JS runtimes and provides background on the components
     used by \natisand to build the sandbox. 
    \item Section~\ref{sect:sci-ffi} motivates the need of
     ad hoc access restrictions for native code highlighting the
     security implications of an otherwise overpermissive behavior.
    \item Section~\ref{sect:design-and-implementation} presents
     \natisand objectives and architecture.
    \item Section~\ref{natisand:sect:policy} details the policy
     structure and explains how to generate its permission rules.
    \item Section~\ref{sect:case-study-deno} showcases the achievement
     of the aforementioned objectives by integrating \natisand into
     the Deno runtime. 
    \item Section~\ref{natisand:sect:exp} presents the ability of
     \natisand to mitigate real-world vulnerabilities while
     introducing only a negligible overhead compared to a permissive
     scenario.
    \item Section~\ref{natisand:rel_works} discusses the major
     differences between \natisand and other literature proposals.
    \item Section~\ref{natisand:sect:conclusions} concludes the chapter.
\end{compactitem}
\medskip

\paragraph*{Chapter~\ref{chap:conclusions}} draws the conclusions of
the thesis and discusses future work.

\section{Publications}

This section presents the list of publications authored during the
Ph.D. course that set the basis for this thesis.

\subsubsection*{Articles in journals}

\begin{itemize}
    \nocite{scalable-mondrian}
    \item Sabrina De Capitani di Vimercati, Dario Facchinetti, Sara
        Foresti, Giovanni Livraga, Gianluca Oldani, Stefano
        Paraboschi, Matthew Rossi, Pierangela Samarati.
        ``\textbf{Scalable Distributed Data Anonymization for Large
        Datasets}''. IEEE Transactions on Big Data (TBD), Volume 9,
        Issue 3. IEEE, 2022.
    
    \nocite{k-flat}
    \item Sabrina De Capitani di Vimercati, Dario Facchinetti, Sara
        Foresti, Gianluca Oldani, Stefano Paraboschi, Matthew Rossi,
        and Pierangela Samarati. ``\textbf{Multi-Dimensional Flat
        Indexing for Encrypted Data}''. Under submission.
\end{itemize}

\subsubsection*{Papers in proceedings of international conferences}

\begin{itemize}
    \nocite{dffoprs-percom21}
    \item Sabrina De Capitani di Vimercati, Dario Facchinetti, Sara
        Foresti, Gianluca Oldani, Stefano Paraboschi, Matthew Rossi,
        and Pierangela Samarati. ``\textbf{Scalable distributed data
        anonymization}''. {\em Proceedings of the 2021 IEEE
        International Conference on Pervasive Computing and
        Communications Workshops and other Affiliated Events (PerCom
        Workshops)}. IEEE, 2021.
    
    \nocite{dffoprs-percom21-artifact}
    \item Sabrina De Capitani di Vimercati, Dario Facchinetti, Sara
        Foresti, Gianluca Oldani, Stefano Paraboschi, Matthew Rossi,
        and Pierangela Samarati. ``\textbf{Artifact: Scalable
        distributed data anonymization}''. {\em Proceedings of the
        2021 IEEE International Conference on Pervasive Computing and
        Communications Workshops and other Affiliated Events (PerCom
        Workshops)}. IEEE, 2021.

    \nocite{seapp}
    \item Matthew Rossi, Dario Facchinetti, Enrico Bacis, Marco Rosa,
        and Stefano Paraboschi. ``\textbf{SEApp: Bringing Mandatory
        Access Control to Android Apps.}''. {\em Proceedings of the
        30th USENIX Security Symposium (USENIX Security 21)}. USENIX,
        2021.

    \nocite{ityt}
    \item Enrico Bacis, Dario Facchinetti, Marco Guarnieri, Marco
        Rosa, Matthew Rossi, and Stefano Paraboschi. ``\textbf{I \em
        Told You Tomorrow: Practical Time-Locked Secrets using Smart
        Contracts}''. {\em Proceedings of the 16th International
        Conference on Availability, Reliability and Security (ARES)}.
        ACM, 2021.

    \nocite{dffoprs-globecom2021}
    \item Sabrina De Capitani di Vimercati, Dario Facchinetti, Sara
        Foresti, Gianluca Oldani, Stefano Paraboschi, Matthew Rossi,
        and Pierangela Samarati. ``\textbf{Multi-dimensional indexes
        for point and range queries on outsourced encrypted data}''.
        {\em Proceedings of the 2021 IEEE Global Communications
        Conference (GLOBECOM)}. IEEE, 2021.

    \nocite{cage4deno}
    \item Marco Abbadini, Dario Facchinetti, Gianluca Oldani, Matthew
        Rossi, and Stefano Paraboschi. ``\textbf{Cage4Deno: A
        Fine-Grained Sandbox for Deno Subprocesses}''. {\em
        Proceedings of the 2023 ACM Asia Conference on Computer and
        Communications Security (ASIACCS)}. ACM, 2023.

    \nocite{enhance-wasm-sandbox}
    \item Marco Abbadini, Michele Beretta, Dario Facchinetti, Gianluca
        Oldani, Matthew Rossi, and Stefano Paraboschi.
        ``\textbf{Leveraging eBPF to enhance sandboxing of
        WebAssembly runtimes}''. {\em Proceedings of the 2023 ACM Asia
        Conference on Computer and Communications Security (ASIACCS)}.
        ACM, 2023.
    
    \nocite{natisand}
    \item Marco Abbadini, Dario Facchinetti, Gianluca Oldani, Matthew
        Rossi, and Stefano Paraboschi. ``\textbf{NatiSand: Native Code
        Sandboxing for JavaScript Runtimes}''. {\em Proceedings of the
        26th International Symposium on Research in Attacks,
        Intrusions and Defenses (RAID)}. ACM, 2023.

    \nocite{dmng}
    \item Marco Abbadini, Michele Beretta, Dario Facchinetti, Gianluca
        Oldani, Matthew Rossi, and Stefano Paraboschi.
        ``\textbf{Lightweight Cloud Application Sandboxing}''. Under
        submission.

    \nocite{freya-ipfs}
    \item Marco Abbadini, Michele Beretta, Sabrina De Capitani di
        Vimercati, Dario Facchinetti, Sara Foresti , Gianluca Oldani,
        Stefano Paraboschi, Matthew Rossi, and Pierangela Samarati.
        ``\textbf{Supporting Data Owner Control in IPFS Networks}''.
        Under submission.
\end{itemize}

Over the years operating systems security has greatly evolved and has
been able to address many of the threats originating by an extensive
and varied set of adversaries.

The mitigation of security threats is particularly important for
mobile operating systems, due to their wide deployment and the
confidential information they hold. Focusing on Android, which is open
source and more accessible to researchers, we notice that components
belonging to the same application share access to the app internal
storage and system services. So, third-party code included by
developers in their applications for monetization and
fast-paced-development purposes have the same access to internal data
and system services as the rest of the application. This represents a
threat to the security of applications and the privacy of users. Thus,
we propose \seapp, a mechanism allowing developers to isolate the
internal components of Android apps and regulate their permissions on
a per-component basis. As a natural evolution of the security
mechanisms already available in Android, \seapp design requires to
(i) preserve the security of system components, (ii) limit how it
may affect the development of applications, and (iii) minimize the
performance impact. Our evaluations show our proposal meet these
requirements, and \seapp is a crucial step to provide strong user
privacy guarantees and comply to data privacy regulations despite the
use of third-party code. This claim is supported by the current
direction Google is following for the development of its Privacy
Sandbox on Android~\cite{android-privacy-sandbox}.

With the research conducted on Android it soon became clear that,
while securing Android apps on the user device was very important,
it was as much important to secure the cloud applications those
applications interact with. Indeed, modern cloud applications can
quickly grow to an elaborate and intricate tangle of services, and
keeping track of all the moving parts and their security boundaries
can be very challenging. Nevertheless, in order to mitigate the
impact of security incidents companies need to pay close attention
to the security of their cloud applications. Several research works
and industrial standards recommend the integration of least
privilege policies to prevent disruptions such as file system
tampering. Unfortunately current technologies prove to be overly
coarse to effectively restrict access control of system resources.
To address this problem, we propose an approach that restricts access
to file system resources with a resource-based granularity.
Specifically, we provide an instrumentation solution to retrieve all
the file system resources required by an application component. Then,
we demonstrate how this information can be used to create fine-grained
access policies, and introduce sandboxing using recent kernel security
modules, thus strengthening the security boundary of the whole
application.

Even though effective in restricting access to an entire microservice,
the approach above can be further improved by considering the
specifics of emerging technologies that enable finer-grined
compartmentalization of cloud applications. In the context of
WebAssembly runtimes (e.g., Wasmtime and WasmEdge), we highlight the
security implications of enabling access to system resources through
the WebAssembly System Interface (WASI), and identify opportunities
for improvement. Here, not only our approach permits to introduce
fine-grained policies to restrict file system access, it also replaces
error-prone userspace implementations of the security checks (and the
security issues stemming from them) with a unified eBPF
implementation.

In the context of using JavaScript and TypeScript for the development
of cloud applications, we investigate how JS runtimes secure the
applications they host, and identify a few limitations. Modern
runtimes (i.e., Node.js, Deno and Bun) render JavaScript code in a
secure and isolated environment, however access to system resources
and the execution of native code raise security concerns, since they
break the JS application isolation. Developers commonly rely on native
code to speed up the development and execution of their applications
and, despite that, JS runtimes do not provide built-in solutions for
the isolation of native code, so we propose the introduction of
\natisand; a component for JavaScript runtimes to control the
filesystem, Inter-Process Communication (IPC), and network resources
available to binary programs and shared libraries. Our solution is
characterized by a compact, generic architecture that fits nicely with
modern runtimes. Internally, it leverages {\em
  Seccomp}~\cite{seccompbpf} and Linux Security Modules (LSMs), such
as {\em Landlock}~\cite{landlock} and {\em eBPF}~\cite{corbet2014eBPF}
to restrict access to protected resources.

During the research work, considerable attention was dedicated to the
usability, effectiveness, and performance of our proposals. Indeed, in
most of our proposals, we allow developers to take full advantage of
the high level security mechanisms without requiring them to fully
comprehend the advanced security features employed underneath. The
developer is only required to provide a concise and readable policy
file, which details the access privileges available to the components
of their application and may be automatically generated. Moreover, we
provide extensive evaluations showcasing the mitigation capabilities
of our proposals with respect to common use cases and severe CVEs.
Alongside the security guarantees introduced, we evaluate also their
performance footprint, and can confirm the introduction of very
limited overhead with respect to the original unsecure systems and
significant improvements with respect to alternative approaches. To
promote the reproducibility of our experimental evaluations, and
facilitate the integration of our solutions with existing real systems
our prototype implementations are all available open source.

\section{Document structure}

The thesis is organized in six chapters.

\paragraph*{Chapter~\ref{chap:intro}} illustrates the document
structure and the publications that set the basis for this thesis.

\paragraph*{Chapter~\ref{chap:seapp}} describes \seapp~\cite{seapp}, a
proposal that provides developers with a mechanism to isolate the
internal components of Android apps and regulate their permissions on
a per-component basis. This is achieved by first executing components
in dedicated processes, and then restricting access to the app and
system resources with ad hoc \sel policies. Specifically, it is
possible to declare rules to regulate access to both the internal app
storage and system services, which are otherwise shared between all
the application components, including third-party libraries the app
depends upon. This is a crucial step to provide strong user privacy
guarantees despite relying on third-party code to achieve monetization
and fast-paced development of Android applications. The prototype
implements a patch to the Android Open Source Platform (AOSP) to
demonstrate the feasibility, effectiveness and efficiency of the novel
approach.

\smallskip
\noindent The chapter is organized as follows.

\begin{compactitem}
    \item Section~\ref{sect:seapp_introduction} provides an overview
     of how the security architecture of mobile systems evolved with
     the maturity of the ecosystem.
    \item Section~\ref{sect:seapp_andro_sec} introduces the techniques
     currently enforcing access control in Android.
    \item Section~\ref{sect:seapp_motiv} presents the motivation for
     the introduction of intra-app isolation in Android. Specifically,
     a set of use cases is used to showcase the security measures
     introduced by \seapp.
    \item Section~\ref{sect:seapp_lang} details the \seapp policy
     module syntax, and its constraints to ensure proper integration
     of app and system policies.
    \item Section~\ref{sect:seapp_config} illustrates the policy
     configuration files of \sea and how to use them in \seapp.
    \item Section~\ref{sect:seapp_implementation} discusses the
     changes the \seapp implementation introduced in Android platform.
    \item Section~\ref{sect:seapp_performance} presents the
     experimental evaluation, in which we measure both the
     installation time and runtime overhead introduced by \seapp.
    \item Section~\ref{sect:seapp_relwork} discusses the major
     differences between \seapp and other literature proposals.
    \item Section~\ref{sect:seapp_conclusions} concludes the chapter.
\end{compactitem}
\medskip

\paragraph*{Chapter~\ref{chap:dmng}} highlights the limitations of
the cloud technologies available at the time of writing with regard to
fine-grained access control of system resources. Then, it addresses
the problem proposing an approach that restrict application access to
file system resources with a resource-based granularity. To this end,
we develop \dmng, a flexible and intuitive tool that relies on
instrumentation to collect and audit the activity traces generated by
microservices. Finally, we demonstrate how this information can be
used to create fine-grained access control policies and strengthen the
security boundary of the cloud application.

\smallskip
\noindent The chapter is organized as follows.
\begin{compactitem}
    \item Section~\ref{dmng:sect:introduction} presents the challeges
     of securing cloud applications and briefly discusses the state of
     the art to identify opportunities for improvement.
    \item Section~\ref{sect:motivation} discusses the objectives of
     the solution together with its requirements and trust
     assumptions.
    \item Section~\ref{sect:overview} illustrates an overview of the
     approach by highlighting its integration with both staging and
     production environments.
    \item Section~\ref{sect:cloud-instrum} presents two different
     techniques to instrument cloud applications and trace their
     activity.
    \item Section~\ref{dmng:sect:policy} discusses the generation of
     access control policies from activity traces.
    \item Section~\ref{sect:sandbox} details how we implement the
     lightweight enforcement of the policies.
    \item Section~\ref{dmng:sect:exp} presents an empirical
     evaluation of the mitigation capabilities of our approach and
     its performance.
    \item Section~\ref{dmng:sect:related-work} discuss proposals from
     the literature and how they compare to ours.
    \item Section~\ref{dmng:sect:conclusions} concludes the chapter.
\end{compactitem}
\medskip

\paragraph*{Chapter~\ref{chap:wasm}} explores the use of WebAssembly
outside of web browsers thanks to WebAssembly runtimes (e.g., Wasmtime
and WasmEdge). Specifically, it highlights the security implications
of enabling access to system resources through the WebAssembly System
Interface (WASI) and it identifies opportunities for improvement. With
specific regard to file system resources, runtimes must prevent
hostcalls (i.e., functions provided by the runtime to the guest
WebAssembly application) from accessing arbitrary locations. Current
implementations introduce security checks to only permit access to a
predefined list of directories. This approach not only suffers from
poor granularity, it is also error-prone and has led to security
issues. In this chapter we propose to replace the security checks in
hostcall wrappers with eBPF programs, enabling the introduction of
fine-grained per-module policies.

\smallskip
\noindent The chapter is organized as follows.
\begin{compactitem}
    \item Section~\ref{intro} introduces WebAssembly and the
     challenges WebAssembly runtimes are facing with restricting
     access to system resources.
    \item Section~\ref{sect:wasm:threat-model} models the threat
     an attacker can pose to the implementations of WASI in
     WebAssembly runtimes.
    \item Section~\ref{design} details the design of current
     WASI implementations and proposes an alternative approach.
    \item Section~\ref{sec:exp} evaluates the overhead introduced by
     our novel approach in two different WebAssembly runtimes.
    \item Section~\ref{wasm:rel_works} discusses the related work.
    \item Section~\ref{sect:wasm:conclusions} concludes the chapter.
\end{compactitem}
\medskip

\paragraph*{Chapter~\ref{chap:natisand}} considers the use of
JavaScript (JS) and TypeScript (TS) for implementing the backend of
cloud applications. In this setting, the execution of JS code on the
server-side is enabled by JS runtimes (i.e., Node.js, Deno and Bun).
These sophisticated software securely renders JS code in an isolated
sandbox, however access to system resources and the execution of
native code raise security concerns, since they effectively break the
isolation between the JS application and the host OS. Identifying these
clear security issues, this chapter proposes \natisand, a component for
JavaScript runtimes that leverages {\em Landlock}, {\em eBPF}, and
{\em Seccomp} to control the filesystem, Inter-Process Communication
(IPC), and network resources available to binary programs and shared
libraries. We demonstrate the effectiveness and efficiency of our
approach by implementing and integrating it into Deno, a modern,
security-oriented JavaScript runtime.

\smallskip
\noindent The chapter is organized as follows.
\begin{compactitem}
    \item Section~\ref{sect:introduction} presents the development of
     web applications with the use of JavaScript runtimes and their
     limitations with regard to controlling native code access to
     system resources.
    \item Section~\ref{sectbackground} overviews the structure of
     moderm JS runtimes and provides background on the components
     used by \natisand to build the sandbox.
    \item Section~\ref{sect:sci-ffi} motivates the need of
     ad hoc access restrictions for native code highlighting the
     security implications of an otherwise overpermissive behavior.
    \item Section~\ref{sect:design-and-implementation} presents
     \natisand objectives and architecture.
    \item Section~\ref{natisand:sect:policy} details the policy
     structure and explains how to generate its permission rules.
    \item Section~\ref{sect:case-study-deno} showcases the achievement
     of the aforementioned objectives by integrating \natisand into
     the Deno runtime.
    \item Section~\ref{natisand:sect:exp} presents the ability of
     \natisand to mitigate real-world vulnerabilities while
     introducing only a negligible overhead compared to a permissive
     scenario.
    \item Section~\ref{natisand:rel_works} discusses the major
     differences between \natisand and other literature proposals.
    \item Section~\ref{natisand:sect:conclusions} concludes the chapter.
\end{compactitem}
\medskip

\paragraph*{Chapter~\ref{chap:conclusions}} draws the conclusions of
the thesis and discusses future work.

\section{Publications}

This section presents the list of publications authored during the
Ph.D. course that set the basis for this thesis.

\subsubsection*{Articles in journals}

\begin{itemize}
    \nocite{scalable-mondrian}
    \item Sabrina De Capitani di Vimercati, Dario Facchinetti, Sara
        Foresti, Giovanni Livraga, Gianluca Oldani, Stefano
        Paraboschi, Matthew Rossi, Pierangela Samarati.
        ``\textbf{Scalable Distributed Data Anonymization for Large
        Datasets}''. IEEE Transactions on Big Data (TBD), Volume 9,
        Issue 3. IEEE, 2022.
    
    \nocite{k-flat}
    \item Sabrina De Capitani di Vimercati, Dario Facchinetti, Sara
        Foresti, Gianluca Oldani, Stefano Paraboschi, Matthew Rossi,
        and Pierangela Samarati. ``\textbf{Multi-Dimensional Flat
        Indexing for Encrypted Data}''. Under submission.
\end{itemize}

\subsubsection*{Papers in proceedings of international conferences}

\begin{itemize}
    \nocite{dffoprs-percom21}
    \item Sabrina De Capitani di Vimercati, Dario Facchinetti, Sara
        Foresti, Gianluca Oldani, Stefano Paraboschi, Matthew Rossi,
        and Pierangela Samarati. ``\textbf{Scalable distributed data
        anonymization}''. {\em Proceedings of the 2021 IEEE
        International Conference on Pervasive Computing and
        Communications Workshops and other Affiliated Events (PerCom
        Workshops)}. IEEE, 2021.
    
    \nocite{dffoprs-percom21-artifact}
    \item Sabrina De Capitani di Vimercati, Dario Facchinetti, Sara
        Foresti, Gianluca Oldani, Stefano Paraboschi, Matthew Rossi,
        and Pierangela Samarati. ``\textbf{Artifact: Scalable
        distributed data anonymization}''. {\em Proceedings of the
        2021 IEEE International Conference on Pervasive Computing and
        Communications Workshops and other Affiliated Events (PerCom
        Workshops)}. IEEE, 2021.

    \nocite{seapp}
    \item Matthew Rossi, Dario Facchinetti, Enrico Bacis, Marco Rosa,
        and Stefano Paraboschi. ``\textbf{SEApp: Bringing Mandatory
        Access Control to Android Apps.}''. {\em Proceedings of the
        30th USENIX Security Symposium (USENIX Security 21)}. USENIX,
        2021.

    \nocite{ityt}
    \item Enrico Bacis, Dario Facchinetti, Marco Guarnieri, Marco
        Rosa, Matthew Rossi, and Stefano Paraboschi. ``\textbf{I \em
        Told You Tomorrow: Practical Time-Locked Secrets using Smart
        Contracts}''. {\em Proceedings of the 16th International
        Conference on Availability, Reliability and Security (ARES)}.
        ACM, 2021.

    \nocite{dffoprs-globecom2021}
    \item Sabrina De Capitani di Vimercati, Dario Facchinetti, Sara
        Foresti, Gianluca Oldani, Stefano Paraboschi, Matthew Rossi,
        and Pierangela Samarati. ``\textbf{Multi-dimensional indexes
        for point and range queries on outsourced encrypted data}''.
        {\em Proceedings of the 2021 IEEE Global Communications
        Conference (GLOBECOM)}. IEEE, 2021.

    \nocite{cage4deno}
    \item Marco Abbadini, Dario Facchinetti, Gianluca Oldani, Matthew
        Rossi, and Stefano Paraboschi. ``\textbf{Cage4Deno: A
        Fine-Grained Sandbox for Deno Subprocesses}''. {\em
        Proceedings of the 2023 ACM Asia Conference on Computer and
        Communications Security (ASIACCS)}. ACM, 2023.

    \nocite{enhance-wasm-sandbox}
    \item Marco Abbadini, Michele Beretta, Dario Facchinetti, Gianluca
        Oldani, Matthew Rossi, and Stefano Paraboschi.
        ``\textbf{Leveraging eBPF to enhance sandboxing of
        WebAssembly runtimes}''. {\em Proceedings of the 2023 ACM Asia
        Conference on Computer and Communications Security (ASIACCS)}.
        ACM, 2023.
    
    \nocite{natisand}
    \item Marco Abbadini, Dario Facchinetti, Gianluca Oldani, Matthew
        Rossi, and Stefano Paraboschi. ``\textbf{NatiSand: Native Code
        Sandboxing for JavaScript Runtimes}''. {\em Proceedings of the
        26th International Symposium on Research in Attacks,
        Intrusions and Defenses (RAID)}. ACM, 2023.

    \nocite{dmng}
    \item Marco Abbadini, Michele Beretta, Dario Facchinetti, Gianluca
        Oldani, Matthew Rossi, and Stefano Paraboschi.
        ``\textbf{Lightweight Cloud Application Sandboxing}''. {\em
        Proceedings of the 14th IEEE International Conference on Cloud
        Computing Technology and Science (CLOUDCOM)}. IEEE, 2023.

    \nocite{freya-ipfs}
    \item Marco Abbadini, Michele Beretta, Sabrina De Capitani di
        Vimercati, Dario Facchinetti, Sara Foresti , Gianluca Oldani,
        Stefano Paraboschi, Matthew Rossi, and Pierangela Samarati.
        ``\textbf{Supporting Data Owner Control in IPFS Networks}''.
        {\em Proceedings of the 2024 IEEE International Conference on
        Communications (ICC)}. IEEE, 2024.
\end{itemize}

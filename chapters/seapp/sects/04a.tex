\section{Policy language} \label{sect:seapp_lang}

To support the use cases presented in Section~\ref{sect:seapp_motiv},
we want the developer to have control of the \sel security context of
subjects and objects related to her security enhanced app.  To each of
them is assigned a {\em type} (also called {\em domain} when it labels
processes).  As types directly relate to groups of permissions, the
evaluation of security contexts is the foundation of each security
decision.  Since apps may offer many complex functions, the policy
language has to provide the flexibility of defining multiple domains
with distinct privileges so that the app, according to the task it has
to do, may switch to the least privileged domain needed to accomplish
the job.

\begin{table*}[h!]
	\centering
        \small
	\begin{tabularx}{\textwidth}{|p{5.8em}p{0.15em}X|}
	\hline
	\multicolumn{3}{|l|}{\textbf{Policy module syntax}} \\ \hline
	\textit{blockStmt} & $\to$& $($\texttt{block} \textit{blockId} \textit{cilStmt}$*$$)$ \\

	\textit{cilStmt} & $\to$&  \textit{ typeStmt} $|$
                                  \textit{typeAttrStmt} $|$
                                  \textit{typeAttrSetStmt} $|$
                                  \textit{typeBoundsStmt} $|$
                                  \textit{typeTransStmt} $|$
                                  \textit{macroStmt} $|$
                                  \textit{allowStmt} \\

	\textit{typeStmt} & $\to$ & $($\texttt{type} \textit{typeId}$)$ \\

	\textit{typeAttrStmt} & $\to$& $($\texttt{typeattribute} \textit{typeAttrId}$)$ \\

	\textit{typeAttrSetStmt} & $\to$ & $($\texttt{typeattributeset} \textit{typeAttrId} $($$\langle$\textit{typeId} $|$ \textit{typaAttrId$\rangle$}$+$$)$$)$ \\

	\textit{typeBoundsStmt} & $\to$& $($\texttt{typebounds} \textit{parentTypeId} \textit{childTypeId}$)$ \\

	\textit{typeTransStmt} & $\to$ &  $($\texttt{typetransition} \textit{sourceTypeId} \textit{targetTypeId} \textit{classId} $[$\textit{objectName}$]$ \textit{defaultTypeId}$)$ \\

	\textit{macroStmt} &$\to$ &$($\texttt{call} \textit{macroId} $($\textit{typeId}$)$$)$ \\

	\textit{allowStmt} & $\to$& $($\texttt{allow} $\langle$\textit{sourceTypeId} $|$ \textit{sourceTypeAttrId}$\rangle$  $\langle$\textit{targetTypeId} $|$ \textit{targetTypeAttrId} $|$ \texttt{self}$\rangle$  \textit{classPermissionId}$+$$)$ \\ \hline

      \end{tabularx}
        \caption{\bf Application policy module CIL syntax}      
  \label{tab:seapp_syntax}
\end{table*}


The app policy is specified in a module, provided by the app to
describe its own types.  The policy module is processed at app
installation time by a component of the system, called {\em \pap
  Policy Parser}, responsible to verify that the policy is correct and
does not introduce vulnerabilities into the system.  The addition of a
policy module is managed by combining the new module with the platform
policy and the previous installed ones, producing after policy
compilation a single binary representation of the global policy.

In this section, we provide a description of the \pap policy language
and the restrictions each module is subject to.  Policy configuration
is detailed in Section~\ref{sect:seapp_config}, while policy
compilation and runtime support are discussed in
Section~\ref{sect:seapp_implementation}.

\subsection{Choice of policy language}\label{seapp_lang}

\sea supports two languages for policies, Type Enforcement
(TE)~\cite{seapp_libte_rules} and Common Intermediate Language
(CIL)~\cite{seapp_cil}.  TE was the language available in the early
implementations of \sel, while CIL was later introduced to offer an
easy to parse syntax that avoids the pervasive use of general purpose
macro processors (e.g., M4~\cite{seapp_m4}).  Another aspect that
differentiates them is that, in Android, TE representations are
internally converted into CIL before being compiled into the \sel
binary policy.  To avoid the additional translation step being
performed at each policy module installation, we decided to use CIL
over TE.

\subsection{Definition of types and type-attributes}

CIL offers a multitude of commands to define a policy, but only a
subset has been selected for the definition of an app policy module.
This was done to control the impact of the policy module on the system
and it may, as a side effect, facilitate the work of the app developer
writing the policy.

The syntax is described in Table \ref{tab:seapp_syntax}.  To declare a
\emph{type}, the \texttt{type} statement can be used.  This permits to
declare the types involved in an access vector (AV) rule, which grants
to a source type a list of permissible actions over a target type.  AV
rules are defined through the \allow statement.

When writing a policy, there is frequently the need to assign the same
set of authorizations to multiple types.  To avoid the repetition of
multiple \allow declarations, it is convenient to refer to multiple
types using a single entity, the {\em type-attribute}.  Using the
\typeattributeset statement we associate with a \typeattribute a set
of types and type-attributes.  Each type-attribute essentially
represents the set of types that is produced by the (possibly
multi-step) expansion of its definition.  The semantics is that each
of the types that directly or indirectly (using type-attributes)
appears as the source of an allow rule will be authorized to operate
with the specified permission on each of the types directly or
indirectly appearing as the target.  This improves the conciseness and
readability of the policy.

After defining the domains with the least group of permissions
necessary to fulfill the task, the developer can also configure the
domain transitions using the \typetransition statement.
By doing so, it is possible to ensure that important native processes run
in dedicated domains with limited privileges, leading to intra-app
compartmentalization.


\subsection{Policy constraints}\label{subsect:seapp_constraints}

The introduction of dedicated modules for apps raises the need to
carefully consider the integration of apps and system policies.  The
first requirement is that an app policy must not change the system
policy and can only have an impact on processes and resources
associated with the app itself.  To preserve the overall consistency
of the \sel policy, each policy module must respect some constraints.
Since Android supports the side-loading of
apps~\cite{seapp_adb_install}, we cannot rely on app markets to verify
app policies.  Therefore, the enforcement of constraints is done on
the device, by both the \pap Policy Parser and the \sel environment.
If any of these components raises an exception, during the
verification or compilation of the policy, app installation is
stopped.

To ensure that policy modules do not interfere with the system policy
and among each other, a first necessity is that policy modules are
wrapped in a unique namespace obtained from the package name.  This is
done through the {\tt block} CIL statement, which prevents the
definition of the same \sel type twice, as the resulting global
identifier is formed by the concatenation of the namespace and the
local type identifier.  Also, the use of a namespace specific for the
policy module permits to discriminate between local types or
type-attributes $T_A$ (namespace equal to the current app package
name), types or type-attributes of other modules $T_{A'\neq A}$
(namespace equal to some other app package), and system types or
type-attributes $T_S$ (system namespace).  At installation time, the
\pap Policy Parser determines the origin of each type, with an
explicit prohibition for policies to refer to types or type-attributes
defined by other policy modules, while use of system types or
type-attributes is subject to restrictions.


With regard to the {\tt allow} statement, a dedicated analysis is
performed by the \pap Policy Parser.  For each rule, the global origin
of source and target types is determined.  We refer to system origin
{\em S}, when the type is directly or indirectly associated with a
system type in the expansion of its definition, while to local origin
{\em A} otherwise.  Based on the origin of source and target of each
rule, there are four cases.  The case {\em AllowSS}, i.e., a
permission with system origin both as source and target, is
prohibited, as it represents a direct platform policy modification.
The case {\em AllowAA} is always permitted, as it only defines access
privileges internal to the app module.  The cases {\em AllowAS} and
{\em AllowSA} are more delicate.

An {\em AllowAS} originates when a local type needs to be granted a
permission on a system type.  A concrete example is shown in
Section~\ref{sect:seapp_motiv}, where the {\em :media} process needs
access to the {\tt camera\textunderscore service}.  The case cannot be
decided locally by the \pap Policy Parser, therefore it is delegated
to the \sel decision engine during policy enforcement.  This crucial
postponed restriction depends on the constraint that all app types
have to appear in a \typebounds statement~\cite{seapp_typebounds},
which limits the bounded type to have at most the access privileges of
the bounding type.  As Android 11 assigns to generic third-party apps
the \untrustedapp domain, this is the candidate we use to bound the
app types.  If the {\em AllowAS} rule gives to the local type more
privileges than those associated with \untrustedapp, and at runtime
these privileges are used, the \sel decision engine identifies the
policy violation and prohibits the action.

{\em AllowSA} rules are the key to regulate how system components
access internal types.  To be compliant with Android, the local types
introduced by the app policy module must ensure interoperability with
system services crucial to the app lifecycle.  As an example {\em
  Zygote}~\cite{seapp_zygoterfi}, the native service which spawns and
configures new app processes, can only execute processes labeled with
the type-attribute {\tt domain}, which is assigned by default to
\untrustedapp.  However, giving app developers the freedom to directly
define {\em AllowSA} rules would lead to two major issues: (i) the
rules would depend on system policy internals, leading to a solution
with limited abstraction and modularity; (ii) explicit {\em AllowSA}
rules could lead to violations of the security assumptions of a system
service, with the risk of introducing vulnerabilities (e.g., leading
to a {\em confused deputy attack}
\cite{seapp_confused_deputy_attack}).  For these reasons we prohibit
their explicit use.  To limit system types to only those already
dealing with untrusted content and simplifying the policy, we rely on
CIL macros, a set of function-like statements that, when invoked by
the \pap policy module, produce a predefined list of policy
statements.  This approach permits to retain control on the rules
produced, ensuring no violation of the default system policy. Also, it
makes the work of the developer easier, by abstracting away system
policy internal details.  To preserve the interoperability with system
services, third-party app functionality has been broken down into the
CIL macros listed in Table~\ref{tab:seapp_macros}. This list has been
identified looking at the internal structure of the {\tt
  untrusted\_app} domain.  With this design philosophy, the developer
can grant a basic set of permissions to a type (by calling one or more
macros), and then add to it fine-grained authorizations with {\em
  AllowAS} rules.
%
\begin{table}[t!]
	\centering
        \small

	\begin{tabular}{|l|l|}
		\hline
                {\bf Macro}           & {\bf Usage}    \\ \hline
                {\em md\textunderscore appdomain}       & to label app domains \\ \hline
                {\em md\textunderscore netdomain}       & to access network    \\ \hline
                {\em md\textunderscore bluetoothdomain} & to access bluetooth  \\ \hline
                {\em md\textunderscore untrusteddomain} & to get full untrusted app permissions \\ \hline
                {\em mt\textunderscore appdatafile}     & to label app files   \\ \hline
	\end{tabular}
	\caption{\bf \pap macros to grant permissions to local types}
	\label{tab:seapp_macros}
\end{table}

With regard to the \typeattributeset statement, the \pap Policy Parser
uses a verification strategy similar to the one used for allow rules.
First, the global origin of the type-attribute and of the set
expression of types and type-attributes is determined.  All statements
that directly or indirectly relate to system types are blocked.  This
avoids implicit permission propagation from system and local types.

Similarly, for the \typetransition statement, the \pap Policy Parser
verifies the origin of the types involved, with a prohibition for all
the statements that relate to system types, as they may lead to an
escalation of privileges.


%%% Local Variables: 
%%% mode: latex
%%% TeX-master: "../../../main.tex"
%%% reftex-default-bibliography: "../../../bib/biblio.bib"
%%% End:
\section{Related work}\label{sect:seapp_relwork}

In traditional desktop operating systems significant effort has been
spent in retrofitting legacy code for authorization policy enforcement
leveraging MAC.  An approach is to place reference monitor calls to
mediate sensitive access locations through the use of static and
dynamic analysis~\cite{seapp_jaeger_retro, seapp_jaeger_hook}.  An
evolution of this solution is the multi-layer reference
monitor~\cite{seapp_jaeger_web}, in which the MAC policy is enforced
at different levels (e.g., application, OS, Virtual Machine Manager).
Another approach is to identify integrity-violating permissions
through the use of information-flow analysis~\cite{seapp_jaeger_flow}.

Android's open source nature and popularity made it the target of
careful security investigations (e.g., \cite{seapp_sok_android,
  seapp_10.1007/978-3-319-20550-2_15, seapp_4768655,
  seapp_10.5555/2028067.2028088}) and several proposals aiming at
strengthen its security properties.  In the following we discuss the
ones that try to address app isolation and modularity, underlining the
key differences with our methodology.

Our approach presents similarities with {\em Secure Application
  INTeraction (Saint)} proposed by Ongtang et al. in
\cite{seapp_semantically_rich}, in which the authors also try to
address the issue of allowing developers to define policies that can
be verified at both installation time and runtime, to better specify
the permissions for each component of their app.  However, since the
paper has been published in 2010, {\em Saint} could not leverage
SEAndroid~\cite{seapp_seandroid}, which was introduced later, thus the
authors had to define their own Android security middleware, which
would not fit into the current Android
architecture~\cite{seapp_tapsm_m}.

{\em FlaskDroid}~\cite{seapp_flaskdroid} defines a versatile
middleware and kernel layer policy language.  It is based on Userspace
Object Managers (USOMs), which control access to services, intents and
data stored in Content Providers.  However, FlaskDroid does not focus
on intra-app compartmentalization, a central aspect in our proposal.

{\em ASM}~\cite{seapp_asm} and {\em ASF}~\cite{seapp_flaskdroid2}
promote the need for a programmable interface that could serve as a
flexible ecosystem for different security solutions. The generality of
these solutions, however, requires to introduce several changes to the
current Android security model.

{\em AppPolicyModules}~\cite{seapp_apm} is another proposal that
allows app developers to create dedicated policy modules. The authors
focus more on how apps could use SEAndroid to better protect their
resources from the system and from other apps, paying limited
attention to internal compartmentalization.

{\em DroidCap}~\cite{seapp_droidcap} is a recent contribution proposed
by Dawoud and Bugiel, in which the authors propose to replace
Android's UID-based ambient authority (DAC) with per-process Binder
object capabilities.  The proposal is interesting as it permits to
achieve security compartmentalization between different app
components. To introduce capability-based access control on files,
{\em DroidCap} had to integrate {\em Capsicum for
  Linux}~\cite{seapp_capsic} in Android. Overall, DroidCap is a nicely
engineered solution, which shares similar objectives with ours, and
the two could work in parallel as they do not interfere with each
other.  However, as our proposal relies on SELinux and SEAndroid,
which are already part of the Android security framework, our
architecture appears to be more aligned with the natural evolution of
the Android ecosystem.

{\em Boxify}~\cite{seapp_boxify} is a virtualization environment for
Android apps, which could be used to achieve a higher level of privacy
and better control over app permissions. The authors also describe how
their solution could be used to compartmentalize Ads libraries to
reduce the risk of sensible information leakage.  Yet, since the
virtualization environment acts as a mediator between the applications
and the system, it extends the set of trusted components the app has
to rely on.

{\em AFrame}~\cite{seapp_aframe} and {\em
  CompARTist}~\cite{seapp_10.1145/3133956.3134064} propose to
compartmentalize third-party libs from their host app using a separate
process with a dedicated UID. In {\em AFrame} the Android Manifest is
modified with the introduction of library ad-hoc permissions, while
{\em CompARTist} uses compile time app rewriting. Both proposals do
not extend the protection at the MAC level.

To summarize, the main differences that characterize our proposal are:
(i) we propose a natural extension of the role of SELinux to apps
leveraging what is already used to protect the system itself, thus
minimizing the impact on it, and (ii) we empower the developers while
limiting the amount of changes an application must undergo in order to
take advantage of our solution.

%%% Local Variables: 
%%% mode: latex
%%% TeX-master: "../../../main.tex"
%%% reftex-default-bibliography: "../../../bib/biblio.bib"
%%% End:
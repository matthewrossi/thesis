\section{Introduction}
\label{sect:seapp_introduction}

Security in operating systems has greatly evolved and has been able to
address many of the threats originating by an extensive and varied
collection of adversaries.

The mitigation of security threats is particularly important
for {\em mobile operating systems}, due to their wide deployment and
the confidential information they hold.

Both Android and iOS have seen significant investments toward the
realization of advanced security techniques, which have led to a great
increase in the level of protection offered to
users~\cite{seapp_tapsm_m}.  The strength of security and the value of
protected resources is testified, for instance, by the payouts
associated with working exploits in markets like
Zerodium~\cite{seapp_zerodium}, where the payouts for mobile operating
systems are the highest\footnote{At the time of writing, US\$2.5M and
  US\$2M are paid for a zero click solution able to subvert the
  security of Androd and iOS, respectively. }.

A peculiar threat that characterizes mobile operating systems is the
need to balance on one side the high sensitivity of the information,
and on the other hand the need for users to install into the system a
large number of applications (called simply {\em apps} in this domain)
often produced by unknown developers, which may hide malicious
functions.  A first level of protection is offered, both in iOS and
Android, by a preliminary screening of apps before they are made
available on the platform market~\cite{seapp_playprotect} or installed
to a device, but this approach cannot provide a strong guarantee.
Security mechanisms internal to the operating system are needed in
order to constrain the apps to only operate within the boundaries
specified by the device owner at installation time.

The approach used in the design of mobile operating systems considers
as the first requirement the protection of system resources.  Focusing
on Android, which is open source and more accessible to researchers,
we notice a significant evolution in its internal security
architecture.  This architecture is quite rich and consists of many
security measures~\cite{seapp_10.1145/2046707.2046779,seapp_tapsm_m}.
In this environment, we specifically look at the role of SELinux.
SELinux implements the {\em Mandatory Access Control} (MAC) mechanism,
which relies on a system-level policy to declare the operations that a
process can execute over a resource based on the security labels
associated with them.  Compared to classical {\em Discretionary Access
  Control} (DAC), still used in Android in an extensive way, MAC is
more rigid and provides stronger guarantees against unwanted
behaviors.  When SELinux was introduced into Android 4.3 in 2013 (see
Figure~\ref{fig:seapp_mac_evolution}), it used a limited set of system
domains and it was mainly aimed at separating system resources from
user apps.  In the next releases, the configuration of SELinux has
progressively become more complex, with a growing set of domains
isolating different services and resources, so that a bug or
vulnerability in some system component does not lead to a direct
compromise of the whole system.

The introduction of SELinux into Android has been a clear success.
Unfortunately, the stronger protection benefits do not extend to
regular apps which are assigned with a single domain named
\untrustedapp.  Since Android 9, isolation of apps has increased with
the use of categories, which guarantees that distinct apps operate on
separate security contexts.  Our proposal, \pap, builds upon the
observation that giving app developers the ability to apply MAC to the
internal structure of the app would provide more robust protection
against other apps and internal vulnerabilities.

%%% Local Variables: 
%%% mode: latex
%%% TeX-master: "../../../main.tex"
%%% reftex-default-bibliography: "../../../bib/biblio.bib"
%%% End:
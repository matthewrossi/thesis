In this thesis, we presented novel fine-grained access control
techniques to protect resources in mobile and cloud applications.

We opened our dissertation by describing \seapp, a technique for
mitigating security threats in the Android mobile operating system.
\seapp empowers developers with the capability of isolating the
internal components of Android apps and regulate their permissions on
a per-component basis. This crucial step not only limits the impact a
vulnerability has on the app resources, but it also allows to provide
strong user privacy guarantees and meet data privacy regulations
despite the use of third-party code.

The dissertation proceeded considering the importance to also secure
cloud applications. Specifically, we presented an approach to support
the introduction of security policies to restrict the file system
resources available to an application. Compared to virtualization
technologies such as VMs and containers, which are associated with
coarse granularity, we demonstrate that our proposal enables the
introduction of fine-grained, per-resource access rules. Then, we
further explore the topic in the context of WebAssembly runtimes.
Here, not only the approach permits to introduce fine-grained
policies to restrict file system access, it also replaces error-prone
userspace implementations of the security checks (and the security
issues stemming from them) with a unified eBPF implementation.
Finally, we consider the specific use of JavaScript runtimes for the
creation of cloud applications, and how developers commonly rely on
native code to speed up development and execution of the application.
Since JS runtimes do not provide solutions for the isolation of native
code, we propose \natisand, a runtime-agnostic component to control
the filesystem, Inter-Process Communication (IPC), and network
resources available to binary programs and shared libraries.

During our research work, considerable attention was dedicated to the
performance and usability of our proposals. The work presented in this
thesis represent the final result of multiple improving iterations to
ensure small performance footprint and high usability by developers.
Indeed, these are very important aspects since the lower the
performance side-effects and the effort on behalf of the developer to
integrate our solutions with existing applications, the broader will
be the adoption of our security mechanisms. With this regard, to
facilitate the integration of our solutions with existing real systems
and foster the reproducibility of our experimental evaluations, our
prototype implementations are all available open source.

We believe the approaches proposed in this thesis contribute to
improving the state of the art in this domain and support the
evolution toward more secure software platforms.

\section{Future work}

This section concludes the thesis with a discussion on the future work
that can be done in the area of fine-grained access control
technologies to protect resources in mobile and cloud applications.

\noindent{\bf Mobile applications} -- Chapter~\ref{chap:seapp}
describes \seapp, a novel proposal that provides developers with a
mechanism to isolate the internal components of Android apps and
regulate their permissions on a per-component basis. This is
achieved by first executing components in dedicated processes, and
then restricting access to the app and system resources with ad hoc
\sel policies. While effective, the decision to use \sel to constraint
the access of application components impose significant limitations.
Resource-wise, \sel implies the use of process isolation to isolate
different components of the application; this increases CPU and memory
utilization which negatively affect responsiveness and battery life of
the mobile device. Usability-wise, despite our efforts, \sel policies
are hard to audit, author and maintain. Therefore, we consider the
possibility of replacing \sel and process isolation with novel Linux
Security Modules and memory protection techniques an interesting and
promising evolution of our work with the potential to solve both these
limitations and, thus, further promote adoption.

\noindent{\bf Cloud applications} -- Chapter~\ref{chap:dmng},
~\ref{chap:wasm}, and~\ref{chap:natisand} highlight limitations of the
cloud technologies available at the time of writing with regard to
fine-grained access control of system resources. Each chapter provides
its own take on a specific aspect of the problem.
Chapter~\ref{chap:dmng} relies on instrumentation to collect and audit
the activity traces generated by microservices, and then uses this
information to create fine-grained access control policies and
strengthen the security boundary of the cloud application.
Chapter~\ref{chap:wasm} focuses on the security implications of
enabling access to system resources through the WebAssembly System
Interface (WASI), and proposes improvements in the controlling access
to file system resources. Chapter~\ref{chap:natisand} considers the
use of JS runtimes for the implementation of cloud applications, and
proposes \natisand, a component to control the filesystem,
Inter-Process Communication (IPC), and network resources available to
binary programs and shared libraries. Interesting future work could
extend the observability and protection to a wider set of system
resources. For example, the use of memory protection techniques could
restrict the area affected by an attack to the sole exploited thread
without leading to potential compromise of the entire JS runtime.
Finally, given the flexibility and versatility of our designs another
interesting future work would be the adoption of these techniques to
secure other Wasm and JS runtimes, or even other interpreted languages
(e.g., PHP, Python, and Ruby).

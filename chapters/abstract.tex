Over the years operating systems security has greatly evolved and has
been able to address many of the threats originating by an extensive
and varied set of adversaries.

The mitigation of security threats is particularly important for
mobile operating systems, due to their wide deployment and the
confidential information they hold. Focusing on Android, we notice
that components belonging to the same application share access to the
app internal storage and system services. While this may not be an
issue when the developer trusts all the code belonging to their
application, it clearly becomes one when third-party code is included
to achieve monetization and fast-paced development. Thus, we propose
\seapp, a mechanism allowing developers to isolate the internal
components of Android apps and regulate their permissions on a
per-component basis. This is a crucial step to provide strong user
privacy guarantees and meeting data privacy regulations despite the
use of third-party code.

With the research conducted on Android it soon became clear that,
while securing Android apps on the user device was very important,
it was as much important to secure the cloud applications those
applications interact with. Indeed, modern cloud applications can
quickly grow to an intricate tangle of services, and unfortunately
current technologies prove to be overly coarse to effectively restrict
access control of system resources. To address this problem, we propose
an approach that restricts access to file system resources with a
resource-based granularity. Then, we further explore the topic in the
context of WebAssembly runtimes (e.g., Wasmtime and WasmEdge).
Specifically, we highlight the security implications of enabling
access to system resources through the WebAssembly System Interface
(WASI), and identify opportunities for improvement.
Finally, we consider the use of JavaScript (JS) and TypeScript (TS)
for the implementation of cloud applications thanks to JS runtimes
(i.e., Node.js, Deno and Bun). These software securely renders JS code
in an isolated sandbox, however access to system resources and the
execution of native code raise security concerns, since they break the
JS application isolation. To address these security issues, we propose
\natisand, a component for JavaScript runtimes to control the
filesystem, Inter-Process Communication (IPC), and network resources
available to binary programs and shared libraries.

The technologies described in this thesis advance the state of the art
in fine-grained resource protection in mobile and cloud applications.
The implementations have been released under open-source licenses and
can be easily integrated with existing real systems.

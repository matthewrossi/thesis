\section{Introduction}
\label{sect:introduction}

JavaScript (JS) and TypeScript (TS) are popular choices for the
implementation of web applications. This success is motivated by their
flexibility, since both are simple to use for the development of
frontend and backend services, and by the vast ecosystem of open
source packages that are available. For instance, the sole {\em npm
  registry} collects more than 1.3 million
packages~\cite{npm-registry}.

The execution of JS code on the server-side is enabled by a {\em JS
  runtime}.  Since its introduction in 2009, Node.js~\cite{node} has
been the de facto solution selected by developers, but recently
Deno~\cite{deno} and Bun~\cite{bun} have received considerable
attention by the community. While the three platforms provide
distinctive features, they all depend on a key external component,
namely the {\em JS engine}, V8~\cite{v8-site} in the
case of Node.js and Deno, JavaScriptCore~\cite{javascriptcore} in the
case of Bun. The engine is a sophisticated software that securely
renders the JS code in an isolated sandbox. Runtimes extend the engine
providing components to access resources and functions that are not
directly available to the web application from within the
sandbox~\cite{node-api, deno-api}. Prominent examples are the
functions to access the network and to read/write the
filesystem. Runtimes also provide support for the execution of native
code -- i.e., running binary programs installed on the host operating
system and calling functions from the available shared libraries.

The support provided by the runtime for the execution of native code
greatly simplifies the work of the developer building the backend of a
web application. However, the APIs enabling access to system resources
and the execution of native code also raise security concerns, since
they effectively break the isolation between the JS application and
the host OS. The ability to control the resources accessible to a JS
program was indeed one of the reasons that led to the creation of Deno
in 2018~\cite{nodejs-regret}, and the solution identified by the
community was to configure the resources available to an application
with simple permission flags~\cite{deno-permissions}. This change also
influenced the design of Node.js, which introduced a similar
flag-based control model\footnote{Node.js support for creating
  security policies is still experimental as of 2023.} two years
later~\cite{node-permissions}. Unfortunately, while permissions are
effective in restricting access to the JS application, they do not
provide isolation guarantees when native code is
executed,
leaving the host exposed to security breaches~\cite{deno-permissions}.

Previous research~\cite{staicu2021bilingual,npm-malicious-update} has
already shown that frequently JS modules depend on components written
in native languages such as C or C++.  The reuse of existing utilities
permits to take advantage of popular high performance libraries and,
in addition to performance, it minimizes the cost of
development. Notable examples are: the node-sqlite3~\cite{
  node-sqlite3} and deno-sqlite3~\cite{deno-sqlite3} database drivers;
modules to perform image/video conversions, such as
sharp~\cite{sharp}, fluent-ffmpeg~\cite{fluentffmpeg-npm} and
gm~\cite{gm-npm}; OCR engines like Tesseract~\cite{tesseract}; and the
cryptography modules relying on bcrypt~\cite{bcrypt}. The 2022 State
of Open Source Security~\cite{snyk-report} claims that each open
source JS project relies on an average of 174 third-party
dependencies; also, each project is estimated to be affected by 40
vulnerabilities when its dependencies are taken into account. Taking
into consideration that web applications in most cases process
untrusted input, the risk of security incidents is high. For instance,
we identified a sample of 32 high severity CVEs\footnote{The list is
  reported in Table~\ref{table-cve}.} that affect native code used by
popular packages (with 2.6M \mbox{downloads/week}), and allow an
adversary to corrupt the filesystem, perform privilege escalation,
execute arbitrary code, open network connections to exfiltrate data,
etc.

\paragraph*{Our contribution}

We see a security gap in the way modern JS runtimes execute native
code, as neither Node.js, nor Deno, nor Bun sandbox it.  In this chapter
we propose \natisand, a framework to provide strong isolation guarantees
against the execution of native code. In detail, \natisand allows the
developer to control on a native-component basis, access to filesystem, Inter-Process
Communication, and network, effectively reducing the risks coming from
the execution of binary programs and shared libraries. Our solution is
characterized by a compact, generic architecture that fits nicely with
modern runtimes. Internally, it leverages {\em
  Seccomp}~\cite{seccompbpf} and Linux Security Modules (LSMs), such
as {\em Landlock}~\cite{landlock} and {\em eBPF}~\cite{corbet2014eBPF}
to restrict access to protected resources. In the design of our
solution we paid attention to usability by developers; it is not
necessary to have a full understanding of its advanced security
features to use it.  The developer is only required to provide a
concise and readable JSON-formatted policy file, detailing the {\em
  ambient rights} -- i.e., the access privileges available to the
components of the web application that rely on native code. To this
end, we provide the developer a comprehensive and interactive CLI tool
to support policy generation, which, as best practice
suggests, can also be integrated into CI/CD pipelines and run against
a set of test cases~\cite{sandtrap,slimtoolkit}. Another key advantage
of our approach is that it permits to sandbox native code preserving
backward compatibility, namely it does not require to change existing
modules (including third-party dependencies) to leverage the new
security features.

We implemented \natisand and integrated it into Deno. We demonstrate the
security benefits showing how our solution mitigates a number of
recent, high-severity vulnerabilities. We performed an extensive
experimental evaluation to assess its performance. We compare the
overhead introduced by our solution to scenarios in which no native
code sandboxing is performed, and when sandboxing is achieved through
other general purpose, state of the art solutions. The experiments
show that, compared to the alternatives, \natisand exhibits substantial
performance improvements. In addition it also provides an easier
interface that does not require any specific security expertise to be
correctly configured.


%%% Local Variables:
%%% mode: latex
%%% TeX-master: "../main.tex"
%%% End:

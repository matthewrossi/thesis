\section{Policy}
\label{natisand:sect:policy}

In this section we present the structure of the policy file, then we
explain how to generate the permission rules.

\subsection{Policy structure}
\label{sect:sci-ffi-policy-structure}

JS applications executed by \natisand are associated with a policy
file. The policy must be provided by the developer before the
application is run, and to this end, a CLI flag (e.g., {\tt
  native-sandbox}) needs to be added to the JS runtime. The policy
file is formatted in JSON, with the following structure: a policy
defines an array of objects and each object details the permissions
available to a security context.  Within each object, a {\em name} is
used to identify the context, a {\em type} indicates whether the
context applies to an executable, a library, or a function of a
library; the sections {\em fs}, {\em ipc} and {\em net} are used to
configure the corresponding permissions. The structure of the objects
is flexible, and only a {\em name} is required to configure a valid
context. As the policy follows a {\em default-deny} model, a context
that specifies only its name has no permissions at runtime. An excerpt
from a policy file is shown in Listing~\ref{natisand:lst:policy_file} (the
complete example is reported in Appendix~\ref{appendix:curl_policy}),
while a summary of the most relevant policy features is described
next.

\subsubsection*{Name, Type}

The name and type elements are used by \natisand to determine which policy
context must be enforced. The type element can be set to {\tt
  executable} (the default value), {\tt library} or {\tt function}.
At runtime \natisand extracts the absolute path of the native program and function
name, and based on the information available, it identifies the most
selective entry in the policy. This gives the developer the
flexibility to use different policies in case binaries and libraries
have the same basename, or when different functions
from the same library are invoked. Moreover, since absolute
paths are used, this approach ensures symlinks cannot trick the lookup
of the security context. Listing~\ref{natisand:lst:policy_file} shows
a policy that is enforced every time the application runs the curl
program.

\subsubsection*{Fs}

The fs element is used to configure filesystem-related permissions. Fs
stores three optional arrays: {\em read}, {\em write} and {\em
  exec}. Filesystem paths are used as array values. As an example, the
context detailed in Listing~\ref{natisand:lst:policy_file} can read and execute
the curl binary, and write to {\tt response.json} in the current
working directory of the web application. In case the developer wants
to operate with a coarser granularity, the value {\tt true} can be
used to replace any of the {\em fs}, {\em read}, {\em write} and {\em
  exec} arrays to grant access to the whole filesystem.

\subsubsection*{Ipc}

To restrict IPC access we decided to expose developers a simple
interface where flags can be turned on and off based on their needs.
Six optional flags are available in the policy: {\em fifo}, {\em
  message}, {\em semaphore}, {\em shmem}, {\em signal}, and {\em
  socket}. For example, in Listing~\ref{natisand:lst:policy_file}, curl is
allowed to use abstract, named, and unnamed Unix sockets. It is up to our sandboxer to abstract away the complexity
of the underlying architecture and enforce the policy when IPC is
performed between groups of threads associated with separate
contexts. No understanding of the standards available (e.g., System V,
POSIX) is required by developers to restrict the permissions
associated with their application. Similarly to the filesystem case, the
developer can use a coarser granularity by setting the {\em ipc}
element to {\tt true}, enabling all communication mechanisms.
Notice that globally available IPC channels are often bound to
filesystem resources, so, while the granularity of the six flags
described above may seem coarse, finer-grained permissions can be
specified leveraging the path associated with the IPC resource.
For example, the developer can restrict the use of a specific named
pipe (pinned to the filesystem) by using its fully qualified path.

\subsubsection*{Net}

Web application developers are often interested in restricting the
hosts an application can connect to. The policy permits to
specify an array of reachable hosts. Each host is fully qualified by
its URL/IP, and the sequence of permitted ports. As in the case of the
filesystem, the policy permits to grant access to the network without
limitations (setting {\em net} to {\tt true}), enable all the ports
for a specific host (setting {\em ports} to {\tt true}), or completely
remove access to the network (leveraging the default-deny
behavior). In Listing~\ref{natisand:lst:policy_file}, the process executing
curl is only allowed to connect to {\tt https://www.example.com} on
port 443.

\begin{lstlisting}[
    abovecaptionskip=-5pt,
    caption=Example of JSON policy file with single context,
    float=tp,
    floatplacement=tbp,
    language=json,
    label=natisand:lst:policy_file,
]
[{
   "name": "/usr/bin/curl",
   "type": "executable",
   "fs": {
           "read":   ["/usr/bin/curl", ...],
           "write":  ["response.json"],
           "exec":   ["/usr/bin/curl", ...]
   },
   "ipc": {
           "socket": true,
   },
   "net": [{
	   "name":   "https://www.example.com",
	   "ports":  [443]
   }]
}]
\end{lstlisting}


\subsection{Policy generation}
\label{sect:sci-ffi-policy-generation}

While designing \natisand we opted for a minimal and easy to understand
policy syntax to target a broad spectrum of users. However, writing a
policy for large components may be a tedious and tricky task, since we
do not expect all the developers to be aware of how binaries and
external libraries used by their web application work internally. To
assist the developer, we follow an approach similar to
SlimToolkit~\cite{slimtoolkit}, where a service is run against a test
suite to generate the security profile.
Specifically, we developed a CLI utility written in Go that provides
generation of policy templates the developer can understand,
modify, and audit. The utility persists policy-relevant information
in a SQLite database, and exposes to the user many functions that
permit to configure multiple contexts, merge the results collected
from multiple tests, and refine policies interactively.
In the following we provide details on
the work we performed for each of the protected subsystems.

\subsubsection*{Filesystem}

The automatic retrieval of the dependencies of a binary is a
well-known problem. Our utility is capable to discover the
dependencies of programs installed as ELF files, and programs that are
spawned by dedicated POSIX or shell wrappers. The utility first uses
{\em ldd}~\cite{linux-ldd} to discover the direct and transient
dependencies, then, it relies on {\em strace}~\cite{linux-strace} to
monitor the interaction between the kernel and a traced binary, so to
complement the information previously found with additional
filesystem permissions.

\subsubsection*{IPC}

\natisand adopts a policy language that abstracts away IPC complexity.  Our
utility supports policy generation by analyzing the results of
multiple test cases where the Seccomp filter and eBPF
programs of the sandboxer are set to {\em auditing mode}.  These
programs return the flags to be enabled.

\subsubsection*{Network}

Network rules are relatively easy to write. However, we do not
assume developers to be necessarily aware of every network connection
needed by the native code. So, we automate the generation of the
policy by observing the execution of the binary with eBPF programs.
In fact, these provide a list of the domain names resolved, their IPs,
and those hardcoded IPs the utility connects to without performing
name resolution. To track domain name resolutions we attach a {\tt 
  uprobe} to the {\tt getaddrinfo} function of the libc library,
for IPs we observe network socket connections using {\tt kprobes} on
{\tt socket\_bind} and {\tt socket\_connect} LSM hooks.
%
 To handle IP address migrations that may occur at runtime,
  we similarly propose to capture the list of IPs returned by the {\tt
    getaddrinfo} with a dedicated eBPF program, which also updates the
  eBPF map of allowed hosts accordingly. By doing so we make sure that
  the security checks reflect the policy. This approach can also be
  extended monitorng DNS traffic on port 53, hence providing support
  for native components that do not rely on libc functions.  

\vspace{2 mm}
\noindent Policy generation is subject to limitations: (i) policy
generation for malicious code 
produces overly permissive policy and obviously cannot be trusted,
(ii) test suite with limited coverage might provide overly strict
policies not allowing the execution of legitimate code.
Overall, the correctness of the policy depends on the test suite used
to collect the permissions. The closer the tests align with the use of
the native utility in production, the more the developer can consider
the policy generated effective. Nonetheless, to have complete
assurance that the policy generated is correct, an auditing process
may be required.

%%% Local Variables:
%%% mode: latex
%%% TeX-master: "../main.tex"
%%% End:

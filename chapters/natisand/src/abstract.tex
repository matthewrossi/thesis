Modern runtimes render JavaScript code in a secure and isolated
environment, but when they execute binary programs and shared
libraries, no isolation guarantees are provided. This is an important
limitation, and it affects many popular runtimes including Node.js,
Deno, and Bun~\cite{node-permissions,deno-permissions}.

In this chapter we propose \natisand, a component for JavaScript runtimes
that leverages {\em Landlock}, {\em eBPF}, and {\em Seccomp} to
control the filesystem, Inter-Process Communication (IPC), and network
resources available to binary programs and shared libraries.  \natisand
does not require changes to the application code and offers to the
user an easy interface.
%
To demonstrate the effectiveness and efficiency of our approach we
implemented \natisand and integrated it into Deno, a modern,
security-oriented JavaScript runtime. We reproduced a number of
vulnerabilities affecting third-party code, showing how they are
mitigated by \natisand. We also conducted an extensive experimental
evaluation to assess the performance, proving that our approach is
competitive with state of the art code sandboxing solutions. The
implementation is available open source.

%%% Local Variables:
%%% mode: latex
%%% TeX-master: "../main"
%%% End:

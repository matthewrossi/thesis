\section{Conclusions}

The increase in scale and complexity of modern web applications has
led to the introduction of new security mechanisms in JS
runtimes. Unfortunately, native code execution still represents
a clear risk, since no isolation is provided by all the major
platforms. \natisand solves this problem, introducing new measures to
confine the execution of binaries and shared libraries. The
proposal is not dependent on a particular JS runtime, and was designed
to be integrated into different architectures.  Considerable
attention was dedicated to usability; little effort is required by
developers to sandbox their applications. Indeed, no specific security
expertise is necessary to benefit from the protection, nor
are changes to the application.

We believe that the approach proposed in this chapter can contribute to
improve the state of the art in this domain and support the evolution
toward more secure software platforms.

%%% Local Variables:
%%% mode: latex
%%% TeX-master: "../main.tex"
%%% End:

\section{Security motivation}
\label{sect:sci-ffi}

JS runtimes let developers specify the set of access privileges given
to a web application~\cite{deno-permissions, node-permissions}.  This
is possible through a set of configuration flags that, when specified,
allow to constrain how JS code can access system resources.  This is a
significant security improvement compared to the past, where
applications were able to access any underlying system
resource~\cite{staicu2018synode, 10.5555/3361338.3361407}.  However,
these constraints only apply to JS code; any function written in other
languages is executed unconstrained, either through a subprocess or
the use of Foreign Function Interfaces (FFI).  Indeed, native code
does not access system resources using the APIs provided by the JS
runtime and the reference monitor of the JS runtime is
bypassed~\cite{deno-permissions}.

There is a broad variety of applications that rely on the use of
native code.  One well-known example is the use of database drivers;
low latency of queries is crucial to satisfy the constraints on
response time of a web application and a pure JS implementation may
not be able to match them. This led to the development of third-party
modules that depend on the code of shared libraries corresponding to
the required database driver (e.g., libsqlite3.so and
libmysqlclient.so).  To testify the wide adoption of this practice,
popular modules for both Node.js (e.g.,
node-sqlite3) and Deno (e.g.,
deno-sqlite3) report more than 600 thousand
\mbox{downloads/week}. Notably, the deno-sqlite3 module was part of
the official showcase of the performance of Deno when invoking the
native code of a shared library~\cite{deno-ffi-improvements}. Previous
work~\cite{staicu2021bilingual} demonstrated how this module can be
exploited with harmful effects for the web application and the
underlying system.

Database drivers are just one example of how web application
development relies on native code.  Other popular use cases are audio
encoding (e.g., libopus), image processing (e.g., ImageMagick,
libvips), video manipulation (e.g., FFmpeg), optical character
recognition (e.g., Tesseract), and many others.  The native code may
contain vulnerabilities, which may be exploited and lead to a variety
of security violations.

\paragraph{{Filesystem compromise}}
Guaranteeing the integrity and confidentiality of the application host
filesystem is crucial to mitigate risks of data corruption and
exfiltration~\cite{berman1995tron}. As a whole a web application often has access
to many critical resources: databases, executables, private keys, user
confidential files, etc. When native code is executed, it can use the
same privileges of the web application.  In line with the
least privilege principle, the potentially vulnerable components
should be able to access only the files needed to perform their
duties.  Authorizing access only to the needed portions of the
filesystem restricts what can be read, written or run by an attacker,
highly limiting the security risk associated with the presence of
vulnerabilities.

\paragraph{Escalation of privileges}
Another relevant attack surface is the privilege of using the IPC
channels provided by the operating system (e.g., pipes, message
queues, unix sockets). By leveraging IPC, a compromised binary can
establish a communication channel with system components and attempt a
confused deputy attack to achieve privilege escalation on the
host~\cite{shao2016misuse, bui2018man}.  Given the potential of this
attack vector, it is important to limit the scope and set of IPC
channels available to a specific native component only to those strictly
necessary for its benign behavior.

\paragraph{{Malicious network channels}}
Network access is a precious resource that a malicious actor can
leverage during an attack. A significant portion of malicious payloads
open reverse shells to gain control of the victim system over the
network~\cite{bulekov2021saphire}.  In addition, attackers may open
network channels to remotely recover data obtained on the vulnerable
host~\cite{schwarz2020seng}. Restrictions on how a single native component of the
web application can access the network can greatly improve the overall
security of the application. Network access should be forbidden or
restricted only to domains defined by the developer, thus restricting
the ability of adversaries to perform data exfiltration or fetch
malicious payloads.

 \vspace{2 mm}
  \noindent Notice that the JavaScript application may require a
  significant number of privileges to ensure all of its components
  operate as intended. Therefore, the application of sandboxing at
  runtime-level rather than native component-level not only is in
  contrast with the least privilege principle, it also increases the
  attack surface, so the chances of an attack to be successful.  

\subsection{Threat model}
\label{sect:threat-model}

We consider the operating system trusted, although binary utilities
may be malicious due to supply chain attacks, or affected by
vulnerabilities due to incorrect memory management, improper data
validation, etc. Protection against attacks targeting JS code is out
of the scope of our proposal, since we consider JS engines and the
permissions system enforced by JS runtimes able to securely render JS
code. \pap aims to constrain the execution of potentially malicious, or vulnerable,
binary utilities and functions used by JS applications.  This native
code accesses system resources unconstrained by the security
mechanisms offered by the JS runtime, and its actions may
cause severe security breaches. Moreover, the input processed by the
web application is often untrusted and can be unsanitized, due to
errors in the sanitization process, misconfiguration or lack of
awareness by the developer.  Therefore, a malicious actor can exploit
this attack vector by submitting specifically crafted requests
targeting the unconstrained native dependencies of the web
application, compromising the host system. The attack vectors may take
multiple forms, e.g., strings, videos, images, and audio files,
depending on the input provided by the JS application to the
vulnerable components. The goal of our proposal is to mitigate the
security risk by
empowering developers with a way to establish clear security
boundaries for the execution of binary utilities and components
depending on them with a per-native-component granularity.

%%% Local Variables:
%%% mode: latex
%%% TeX-master: "../main.tex"
%%% End:

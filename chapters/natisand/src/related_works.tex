\section{Related work}
\label{natisand:rel_works}

Isolation of software has been widely investigated by both the academic
and industrial communities~\cite{berman1995tron, kim2013practical,
  demarinis2020sysfilter, minijail, sandbox2, firejail,
  bubblewrap,seapp,enhance-wasm-sandbox}.  {\em MBOX}~\cite{
  kim2013practical} features an unprivileged sandboxing mechanism that
prevents a process from modifying the host filesystem by layering the
sandbox filesystem on top of it. The solution is implemented by
interposing syscalls using Seccomp and {\em ptrace}. The use of {\em
  ptrace} required the authors careful attention to avoid the risk of
TOCTOU attacks, moreover it suffers from non-negligible performance
degradation. DeMarinis et al.~\cite{demarinis2020sysfilter} propose
{\em Sysfilter}, a static analysis framework to reduce the attack
surface of the kernel, by restricting with Seccomp the system call set
available to processes.  This approach proves to be effective in
limiting the kernel APIs that can be abused by attackers, but whenever
a system call is necessary for the benign behavior of a program, there
is no way to control with Seccomp the specific instance of the
resource used. {\em BPFBox}~\cite{ findlay2020bpfbox}, {\em
  BPFContain}~\cite{findlay2021bpfcontain}, and {\em
  Snappy}~\cite{belair2021snappy} are security frameworks that provide
confinement of processes and containers with the use of eBPF.  These
solutions highlight the benefit of eBPF by providing simple, efficient,
and flexible confinement of system resources, however these solutions
either require a privileged daemon or require to load kernel modules
to introduce a set of {\em dynamic helpers}.   Often,
industrial sandboxing solutions take advantage of multiple protection
techniques to support process containment. This is the case for {\em
  Minijail}~\cite{minijail}, and {\em Sandbox2}~\cite{ sandbox2}. Some
of the security mechanisms used are: introduction of new user ids,
capabilities restriction, namespace isolation and policy-based Seccomp
filtering. These tools expose a powerful interface meant
to be used by security experts. Similar considerations are shared with
other less mature tools such as {\em Firejail}~\cite{ firejail} and
{\em Bubblewrap}~\cite{bubblewrap}.

Multiple research efforts have studied the use of third-party
components in software products~\cite{RLBox, kirth2022pkru,
  wu2012codejail, cali}. {\em PKRU-Safe}~\cite{kirth2022pkru} proposes an
automated method for preventing the memory corruption of memory-safe
languages due to the interaction with unsafe code. It leverages the
compiler infrastructure to provide hardware-backed memory protection
requiring changes to build files, dependencies, and code (in the form
of code annotations). {\em Codejail}~\cite{wu2012codejail} provides
partial isolation of libraries by spawning a new process and
configuring the necessary communication channels to support tight
memory interactions with the main program. To support the change in
the interactions without modifications to the library code it is
necessary to write a wrapper library.
{\em Cali}~\cite{cali} is a compiler-assisted library
isolation system that compartmentalizes libraries into their own
process, and automates the configuration of the necessary
communication channels by tracking data flow between the program and
the library at link time.
{\em RLBox}~\cite{RLBox} is a
framework to isolate libraries in lightweight sandboxes -- i.e.,
process, Wasm. It facilitates the retrofitting of applications
employing static information flow enforcement and dynamic checks
expressed in the C++ type system. These solutions were designed for
compiled languages, so while some of the concepts are portable to JS
runtimes, the solutions are not easily adapted to this domain.

A recent study by Staicu et al.~\cite{staicu2021bilingual} highlights
how the possibility to invoke native code from scripting languages
undermines the security assumption of applications. They discuss a
methodology to detect misuses of the native extension API and show how
the exploit of these vulnerabilities in npm packages can lead to web
applications compromise. Previous proposals~\cite{
  vasilakis2018breakapp, wyss2022wolf, binwrap} tackle this problem by
providing solutions to isolate the execution of third-party
modules. {\em Wolf at the Door}~\cite{wyss2022wolf} reduces the risk
associated with the installation of npm packages by mediating their
install-time capabilities. It enforces complex user-defined policies
by leveraging AppArmor, hence prohibiting unauthorized access to
confidential files and connections using an LSM that currently cannot
coexist with SELinux and SMACK. {\em BreakApp}~\cite{
  vasilakis2018breakapp} takes advantage of module boundaries to
compartmentalize npm modules in accordance with a set of code
annotations. Modules are isolated with software, process, or container
isolation, and it is possible to configure the visibility of the
application context available to external modules. Process and
container isolation enable the protection of native code, however the
specification of their permissions are beyond the scope of the
proposal.
{\em Cage4deno}~\cite{cage4deno} protects filesystem resources from subprocesses
executed by JavaScript runtimes.
{\em BinWrap}~\cite{binwrap} separates the execution of
third-party components from the rest of the application using distinct
execution threads for different domains of trust. The main focus
of the proposal is prohibiting arbitrary accesses to sensitive data
stored in the memory of the JS runtime by leveraging Intel's MPK/PKU.
\pap is complementary to the above solutions since our goal
is to specify and enforce permissions on native code dependencies of
web applications, rather than providing memory isolation for
untrusted components.

Protecting JS code from being compromised is out of the scope of \pap,
nonetheless, since proposals in this domain and ours both target the
web development audience, our proposal shares some ideas with previous
works in this domain~\cite{vasilakis2021preventing, sandtrap,
  terrace2012javascript, ntousakis2021detecting}. For instance, Ferreira et al.~\cite{
  npm-malicious-update} propose a lightweight permission system providing
per-package on/off switches that limit access to Node.js core modules
(e.g., child\_process, fs, http). By doing so, it can prohibit access
to subprocess, filesystem, and network resources for the JS
code. Similarly, \pap takes care of protecting filesystem, IPC, and
network resources, targeting native code. {\em Mir}~\cite{
  vasilakis2021preventing} is a system preventing the compromise of
the application by third-party modules with the enforcement of
fine-grained RWX permissions on every field of every variable in the
JS context. \pap adopts an equivalent permission model to contain
native code when accessing filesystem resources. Another research work
enforcing security boundaries stated in a policy is {\em
  SandTrap}~\cite{sandtrap}. The approach enforces fine-grained
access control policies on cross-domain interactions between
application code and the third-party modules. The creation of policy
files described by the authors consists in running test suites to
create a policy with acceptable static cross-domain interaction
coverage. We adopt a similar approach in the policy generation of
\pap. Note that, differently from our proposal, solutions protecting
JS code can run in user space, thus they do not limit the portability
of the JS runtime to Linux systems.

%%% Local Variables:
%%% mode: latex
%%% TeX-master: "../main.tex"
%%% End:

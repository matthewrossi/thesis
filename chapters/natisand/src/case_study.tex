\section{Case study: Deno runtime}
\label{sect:case-study-deno}

There are three well-known alternatives for the execution of JS code
on the backend, namely Node.js, Deno, and Bun. As highlighted in
Section~\ref{sect:background-js-runtime}, their architectures have
strong similarities, and NatiSand is designed to be compatible with
all of them (since no assumption is made on specific runtime
components). Nevertheless the integration is not trivial, and it requires
significant engineering effort, therefore we integrated \natisand into only
one of them to demonstrate the achievement of the set objectives
(Section~\ref{subsect:objectives}). In this section we explain our
decision, then we highlight the main architectural changes.

\subsection{Runtime selection}
\label{sect:deno-selection}

Considering (i)~popularity among web developers, (ii)~availability and
support of third-party modules, and (iii)~security-oriented features
provided by the runtime, we selected Deno. Bun is the latest runtime
released, and thus currently the least used by developers. Node.js is nowadays the most
widely used platform, and it offers developers the largest collection
of open source packages. However, Node.js by default does not prevent
JS applications to access system resources, and although it
recently introduced a module-based permission model, the feature is
experimental~\cite{node-permissions}. Deno was instead designed with the protection of
the host as one of its main goals~\cite{nodejs-regret}, thus no access to privileged system
resources is given to JS applications unless the developer
explicitly grants it. Deno provides the
{\em Node Compatibility Mode}~\cite{deno-compatibility-mode}, a
feature enabling the reuse of
code and libraries originally built for Node.js. The availability of
this function permits to
import packages hosted by Deno on {\tt deno.land/x}, as well as
modules published to npm. To conclude, Node.js and Deno prevail
over Bun on popularity and security-oriented features. Node.js wins over Deno on
popularity (but Deno is quickly growing), they are comparable in terms
of third-party modules, and Deno significantly outperforms Node.js on
security oriented-features, leading us to choose Deno.

\subsection{Deno integration}
\label{subsect:deno-integration}

Deno has a modular architecture organized into components. Three of
them are particularly important for \natisand: (i) {\em
  rusty\textunderscore v8}, the package that bridges Deno and the V8
engine implementing the set of bindings to the V8's C++ API, (ii) {\em
  deno\textunderscore core}, which leverages {\em rusty\textunderscore
  v8} to expose the interfaces provided by Deno to the JS application,
and (iii) {\em deno}, which defines the runtime executable together
with the Command Line Interface.

\paragraph{Bootstrap}
%
Shortly after the Deno executable is run, the {\em deno} component is
used to read the permissions granted by the developer via CLI. We
extended this stage to read and parse the policy file specified with
the new {\tt native-sandbox} flag, then we added the permissions
associated with each security context to the {\em global state} stored
by the runtime. To complete the bootstrap phase, we also integrated
the steps to load the necessary eBPF programs and to initialize the
pool of isolated contexts, as explained in Section~\ref{subsect:arch}.

\paragraph{Application runtime}
%
After the JS application is started, the function
calls that cannot be directly handled by V8 are routed to the Deno
runtime through the bindings defined by {\em rusty\textunderscore
  v8}. Each of them is associated with the {\em op\textunderscore
  code}, a unique code identifying the operation to be
performed. The {\em deno\textunderscore core} component receives such
requests, it checks the permissions available from the global state,
and serves them accordingly. We identified requests that
require to execute native code (e.g., {\tt command}, {\tt dlopen},
{\tt run}), and modified {\em deno\textunderscore core} so that they
are restricted by \natisand. The {\em op\textunderscore code} along
with the arguments are used to select the proper security context.


\subsection{Support to fast JS calls}
\label{subsect:fast-api}

In October 2020 V8 announced the support to fast JS
calls~\cite{v8-fast-calls}. The function allows V8 to directly invoke optimized native
functions without leveraging the bindings that connect V8 and the
embedder (e.g., JS runtime). This permits to obtain substantial
performance gains, since native function calls can be resolved in
nanoseconds. 

Deno has introduced unstable support to fast JS calls in July
2022~\cite{deno-v1234}. The change affected the implementation of the
{\em dlopen} API, which is now able to generate an optimized and a
fallback (i.e., standard) execution path for native functions. The
optimized path is triggered only when V8 is actually able to optimize
a symbol, and it entails the execution of code leveraging the fast
call interface. While the optimized path is associated with minimum
overhead, from a security perspective it permits the web application
to execute native code without the mediation of the JS runtime,
invalidating the security reference monitor of Deno.
%
In our prototype we address this Deno security issue offering
developers two alternatives: (i) turn off
the fast call support and safely rely on the execution of sandboxed
native functions with \natisand without any code change, and (ii) enable insecure fast calls but
allow to select the JS functions that need to be isolated with minimal code changes. The second
option permits to take advantage of fast calls when performance is
critical and risks are limited (e.g., arithmetic operations), and at
the same time benefit from the security features \natisand provides. To
this end, we introduced a new API named {\tt Deno.nativeCall()}. The API receives, as first argument, the name of the function
to be sandboxed, along with the list of its arguments.
Listing~\ref{lst:sqlite3_example} shows how to sandbox the functions
from the native database driver {\em sqlite3}.
%
\begin{lstlisting}[
	abovecaptionskip=-5pt,
	caption=Code sandboxing with {\tt nativeCall()},
	float=t,
	floatplacement=tbp,
	label=lst:sqlite3_example,
	language=javascript,
	]
function query(db, stmt) {
    const sqliteDB = new sqlite3.Database(db);
    const query = sqliteDB.prepare(stmt);
    const tuples = query.all();
    sqlite_db.close();
    return tuples;
}

const db = "database.db";  
const stmt = "SELECT * FROM table";
const ts = Deno.nativeCall(query, [db, stmt]);
console.log(ts); // print tuples
\end{lstlisting}

%%% Local Variables:
%%% mode: latex
%%% TeX-master: "../main.tex"
%%% End:
